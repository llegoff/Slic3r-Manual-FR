%!TEX root = Slic3r-Manual.tex

Si le maillage 3D décrit dans le modèle contient des trous, ou les bords ne sont pas alignés (connu comme étant non-manifold), alors Slic3r peut avoir des problèmes de traitement . Slic3r va tenter de résoudre les problèmes, s'il le peut, mais certains problèmes sont hors de sa portée. Si l'application indique que le modèle ne peut pas être tranché correctement alors il ya plusieurs options disponibles, et celles décrites ici sont tous libres au moment de l'écriture.

%%% CONFIGURATION TUNING %%%
{%!TEX root = Slic3r-Manual.tex

\paragraph{Netfabb Studio} % (fold)
\label{par:netfabb_studio}
Netfabb produit une gamme d'applications de mod\'elisation 3D, y compris une version de base gratuite\footnote{http://www.netfabb.com/basic.php}.  Cette version comprend un module de r\'eparation de maillage qui peut aider \`a \'eliminer les diff\'erents probl\`emes rencontr\'es. Les instructions mise \`a jour peuvent \^etre trouv\'es sur le wiki Netfabb\footnote{http://wiki.netfabb.com/Part\_Repair}, ce qui suit est un bref aperçu des \'etapes \`a suivre.

\begin{figure}[H]
\centering
\includegraphics[keepaspectratio=true,width=0.75\textwidth]{working_with_models/netfabb_studio_part_repair.png}
\caption{Netfabb Studio: R\'eparation de mod\`ele.}
\label{fig:netfabb_studio_part_repair}
\end{figure}

\begin{itemize}
	\item Lancer Netfabb studio, et charger le fichier STL qui a un probl\`eme, que ce soit par l'interm\'ediaire du menu \texttt{File} ou par glisser-d\'eposer sur l'espace de travail. Si Netfabb d\'etecte un probl\`eme, il affiche un signe d'alerte rouge dans le coin en bas \`a droite.
	\item Pour ex\'ecuter les scripts de r\'eparation, s\'electionnez la partie et puis cliquez sur la premi\`ere ic\^one d'aide dans la barre d'outils (la croix rouge), ou s\'electionnez dans le menu contextuel \texttt{Extras->Repair Part}.  Cela va ouvrir l'onglet r\'eparation de mod\`ele et de montrer l'\'etat du mod\`ele.
	\item Les onglets \texttt{Actions} et \texttt{Repair scripts} offrent plusieurs scripts de r\'eparation qui peuvent \^etre appliqu\'ees manuellement, mais dans le but de cet aperçu s\'electionez le script \texttt{Automatic repair} corrigera la plupart des probl\`emes.
	\item Le bouton de r\'eparation automatique pr\'esente deux options: par d\'efaut et simples. Choisir par d\'efaut couvrira la plupart des cas. Selectionnez \texttt{execute} pour lancer le script.
	\item Une fois la pi\`ece r\'epar\'ee les r\'eparations doivent \^etre appliqu\'ees en s\'electionnant \texttt{Apply repair}, choisissant s'il faut passer outre la partie existante ou non.
	\item La pi\`ece peut ensuite \^etre export\'e en s\'electionnant \texttt{Export part->As STL} \`a partir du menu contextuel.
	\item Si Netfabb d\'etecte encore que la partie export\'ee contient des erreurs, alors il offrira la possibilit\'e d'appliquer d'autres r\'eparations avant de l'exporter.
	\begin{figure}[H]
	\centering
	\includegraphics[keepaspectratio=true,width=0.75\textwidth]{working_with_models/netfabb_studio_export_part.png}
	\caption{Netfabb Studio: Export de pi\`ece.}
	\label{fig:netfabb_studio_export_part}
	\end{figure}
\end{itemize}
% paragraph netfabb_studio (end)

\paragraph{Netfabb Cloud Service} % (fold)
\label{par:netfabb_cloud_service}
Netfabb accueille \'egalement un service web o\`u un fichier STL peut \^etre t\'el\'echarg\'e pour \^etre v\'erifi\'e et r\'epar\'e\footnote{http://cloud.netfabb.com/}.  

\begin{figure}[H]
\centering
\includegraphics[keepaspectratio=true,width=0.75\textwidth]{working_with_models/netfabb_cloud_services.png}
\caption{Netfabb Cloud Services.}
\label{fig:netfabb_cloud_services}
\end{figure}

\begin{itemize}
	\item Acc\'edez \`a http://cloud.netfabb.com
	\item Choisissez le fichier STL \`a t\'el\'echarger en utilisant le bouton pr\'evu.
	\item Une adresse e-mail doit \^etre donn\'ee pour vous informer quand la prestation est termin\'e.
	\item Choisissez entre les mesures m\'etriques ou imp\'eriales qui doivent \^etre utilis\'es.
	\item Lisez et acceptez les conditions de service, puis cliquez sur \texttt{Upload to Cloud}.
	\item Une fois que le service a analys\'e et r\'epar\'e le fichier, un email est envoy\'e, fournissant le lien de t\'el\'echargement du fichier r\'epar\'e.
\end{itemize}
}
%%% END CONFIGURATION TUNING %%%

\paragraph{FreeCAD} % (fold)
\label{par:freecad}
\index{FreeCAD}

Freecad\footnote{\url{http://sourceforge.net/projects/free-cad}} est un logiciel de CAO, complet et gratuit, qui est livré avec un module de maillage, dans lequel on peut effectuer les réparations d'erreur dans les modèles. Les étapes suivantes décrivent comment un problème dans un fichier de modèle peut être analysé et réparé.

\begin{figure}[H]
\centering
\includegraphics[keepaspectratio=true,width=0.75\textwidth]{working_with_models/freecad_part_repair.png}
\caption{Réparation avec FreeCAD.}
\label{fig:freecad_part_repair}
\end{figure}

\begin{itemize}
	\item Lancer FreeCAD et à partir la page d'accueil choisir \texttt{Working with Meshes}.
	\item Chargez le modèle en le faisant glisser sur l'espace de travail ou par l'intermédiaire du menu \texttt{File}.  Un petit message dans le coin en bas à gauche indique si le modèle semble avoir des problèmes.
	\item Dans le menu choisissez \texttt{Meshes->Analyze->Evaluate \& Repair mesh} pour faire apparaître la boîte de dialogue des options de réparation.
	\item Dans la boîte de dialogue choisir la maille chargée, puis effectuer chaque analyse soit en cliquant sur le bouton \texttt{Analyze} par type de problème, ou sélectionnez \texttt{Repetitive Repair} en bas pour effectuer tous les contrôles. Si un problème  correspondant est détecté le bouton \texttt{Repair} devient actif.
	\item Pour chaque réparation souhaité frappé le bouton \texttt{Repair}.
	\item Il est important d'examiner l'effet que le script de réparation a apportées au modèle.  Il se peut que le script produise des dommages dans le fichier, plutôt que de le réparer, par exemple en retirant triangles importants.
	\item Exporter le modèle réparé par le menu \texttt{Export} ou le menu contextuel.
\end{itemize}
% paragraph freecad (end)
