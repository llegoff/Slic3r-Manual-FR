%!TEX root = Slic3r-Manual.tex
\section{Soutien Slic3r} % (fold)
\label{sec:slic3r_support}

\index{community support}
\index{soutien de la communaut\'e}
\index{Freenode}
\index{IRC}
\index{RepRap}
\index{forums}
\index{website}
\index{site web}
\index{blog}

Une vari\'et\'e de ressources est disponibles pour fournir un soutien pour Slic3r.
\subsection{Wiki et FAQ} % (fold)
\label{sub:wiki_and_faq}
Le wiki fournit de la documentation \`a jour, et une section FAQ qui peuvent aider \`a r\'epondre des questions ou des probl\`emes.
\begin{itemize}
    \item \url{https://github.com/alexrj/Slic3r/wiki/Documentation}
    \item \url{https://github.com/alexrj/Slic3r/wiki/FAQ}
\end{itemize}
% subsection wiki_and_faq (end)

\subsection{Blog} % (fold)
\label{sub:blog}
Conseils, astuces et avis sont publi\'es sur le blog Slic3r.
\begin{itemize}
    \item \url{http://slic3r.org/blog}
\end{itemize}
% subsection blog (end)

\subsection{IRC} % (fold)
\label{sub:irc}

Pr\'esentes sur le serveur \texttt{irc.freenode.net}, les salles de chat suivantes sont souvent remplis de gens qui peuvent fournir une aide en temps r\'eel:
\begin{itemize}
\item \texttt{\#reprap}: Communaut\'e tr\`es active du projet RepRap avec de nombreux utilisateurs de Slic3r.
\item \texttt{\#slic3r}: Salon de discussion Slic3r o\`u les d\'eveloppeurs de Slic3r et les utilisateurs peuvent donner de l'aide.
\end{itemize}

% subsection irc (end)

\subsection{Forum RepRap.org} % (fold)
\label{sub:reprap_org_forum}


Un forum d\'edi\'e \`a Slic3r existe dans les forums RepRap.
\begin{itemize}
    \item \url{forums.reprap.org/list.php?263}
\end{itemize}

% subsection reprap_org_forum (end)

\subsection{Suivi des anomalies} % (fold)
\label{sub:issue_tracker}

Si un bogue a \'et\'e d\'ecouvert dans le logiciel alors une question peut \^etre soulev\'ee dans le suivi d'anomalie (issue tracker) du projet.

\begin{itemize}
    \item \url{github.com/alexrj/Slic3r/issues}
\end{itemize}

\textbf{S'il vous pla\^it} prenez le temps de lire les questions d\'ej\`a poss\'ees pour voir si le probl\`eme a d\'ej\`a \'et\'e soumis. V\'erifiez \'egalement que le probl\`eme est un bogue dans l'application, des questions d'aise \`a l'utilisation ne doivent pas \^etre poss\'ees ici.

Si le bogue semble \^etre non d\'eclar\'ee alors s'il vous pla\^it lire le guide de rapport de bogue avant de soumettre: \url{https://github.com/alexrj/Slic3r/wiki/Quick-guide-to-writing-good-bug-reports}.

% subsection issue_tracker (end)

% section slic3r_support (end)