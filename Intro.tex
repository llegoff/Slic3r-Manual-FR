%!TEX root = Slic3r-Manual.tex

\section{Pr\'esentation} % (fold)
\label{sec:overview}

Slic3r est un outil qui traduit des mod\`eles 3D en instructions qui sont interpr\'et\'ees par une imprimante 3D. Il d\'ecoupe le mod\`ele en couches horizontales et g\'en\`ere les chemins appropri\'es pour les combler.

Slic3r est inclus dans plusieurs logiciels: Pronterface, Repetier-host, ReplicatorG, et peut \^etre utilis\'e comme un programme autonome.

Ce manuel fournira des conseils sur la fa\c{c}on d'installer, configurer et utiliser Slic3r afin de produire d'excellentes impressions.

% section overview (end)


\section{Buts \& Philosophie} % (fold)
\label{sec:goals_philosophy}

Slic3r est un projet original commenc\'e en 2011 par Alessandro Ranellucci (alias Sound), qui a utilis\'e sa connaissance approfondie du langage Perl pour cr\'eer une application rapide et facile \`a utiliser. La lisibilit\'e et la maintenabilit\'e du code sont parmi les objectifs de conception.

Le programme est en cours d'am\'elioration constante, Alessandro et les autres contributeurs du projet, apportent de nouvelles fonctionnalit\'es et des corrections de bogues, de fa\c{c}on r\'eguli\`ere.

% section goals_philosophy (end)


\section{Faire un don} % (fold)
\label{sec:donating}

Slic3r a commenc\'e comme un travail d'un seul homme, d\'evelopp\'e exclusivement par Alessandro \`a ses heures perdues, et en tant que d\'eveloppeur ind\'ependant, ce qui a un co\^ut direct pour lui. En lib\'erant g\'en\'ereusement Slic3r au public en tant que logiciel libre , sous licence GPL, il a permis \`a beaucoup de profiter de son travail.

Il est possible de contribuer par un don. Plus de d\'etails sont disponibles \`a l'adresse: \url{http://slic3r.org/donations}.

% section donating (end)