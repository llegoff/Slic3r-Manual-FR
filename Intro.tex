%!TEX root = Slic3r-Manual.tex

\section{Pr\'esentation} % (fold)
\label{sec:overview}

Slic3r est un outil qui traduit des mod\`eles 3D en instructions interpr\'et\'ees par une imprimante 3D. Il d\'ecoupe le mod\`ele en couches horizontales et g\'en\`ere les chemins appropri\'es pour les combler.

Slic3r est inclus dans plusieurs logiciels: Pronterface, Repetier-host, ReplicatorG, et peut \^etre utilis\'e comme un programme autonome.

Ce manuel fournira des conseils sur la fa\c{c}on d'installer, configurer et utiliser Slic3r afin de produire d'excellentes impressions.

% section overview (end)


\section{Buts \& Philosophie} % (fold)
\label{sec:goals_philosophy}

Slic3r est un projet original commenc\'e en 2011 par Alessandro Ranellucci (alias Sound), qui a utilis\'e sa connaissance approfondie du langage Perl pour cr\'eer une application rapide et facile \`a utiliser. La lisibilit\'e et la maintenabilit\'e du code font partis les objectifs de conception.

Le programme est constamment en cours d'am\'elioration, Alessandro et les autres contributeurs du projet, apportent r\'eguli\`erement de nouvelles fonctionnalit\'es et les corrections de bogues.

% section goals_philosophy (end)


\section{Faire un don} % (fold)
\label{sec:donating}

Slic3r a commenc\'e comme un travail d'un seul homme, d\'evelopp\'e exclusivement par Alessandro \`a ses heures perdues, en tant que d\'eveloppeur ind\'ependant, ce qui a un co\^ut direct pour lui. En lib\'erant g\'en\'ereusement Slic3r au public en tant que logiciel libre , sous licence GPL, il a permis \`a beaucoup de profiter de son travail.

Il est possible de contribuer par un don. Vous trouverez plus de d\'etails \`a l'adresse: \url{http://slic3r.org/donations}.

% section donating (end)

%!TEX root = Slic3r-Manual.tex
\section{Soutien Slic3r} % (fold)
\label{sec:slic3r_support}

\index{community support}
\index{soutien de la communaut\'e}
\index{Freenode}
\index{IRC}
\index{RepRap}
\index{forums}
\index{website}
\index{site web}
\index{blog}

Une vari\'et\'e de ressources est disponibles pour fournir un soutien pour Slic3r.
\subsection{Wiki et FAQ} % (fold)
\label{sub:wiki_and_faq}
Le wiki fournit de la documentation \`a jour, et une section FAQ qui peuvent aider \`a r\'epondre des questions ou des probl\`emes.
\begin{itemize}
    \item \url{https://github.com/alexrj/Slic3r/wiki/Documentation}
    \item \url{https://github.com/alexrj/Slic3r/wiki/FAQ}
\end{itemize}
% subsection wiki_and_faq (end)

\subsection{Blog} % (fold)
\label{sub:blog}
Conseils, astuces et avis sont publi\'es sur le blog Slic3r.
\begin{itemize}
    \item \url{http://slic3r.org/blog}
\end{itemize}
% subsection blog (end)

\subsection{IRC} % (fold)
\label{sub:irc}

Pr\'esentes sur le serveur \texttt{irc.freenode.net}, les salles de chat suivantes sont souvent remplis de gens qui peuvent fournir une aide en temps r\'eel:
\begin{itemize}
\item \texttt{\#reprap}: Communaut\'e tr\`es active du projet RepRap avec de nombreux utilisateurs de Slic3r.
\item \texttt{\#slic3r}: Salon de discussion Slic3r o\`u les d\'eveloppeurs de Slic3r et les utilisateurs peuvent donner de l'aide.
\end{itemize}

% subsection irc (end)

\subsection{Forum RepRap.org} % (fold)
\label{sub:reprap_org_forum}


Un forum d\'edi\'e \`a Slic3r existe dans les forums RepRap.
\begin{itemize}
    \item \url{forums.reprap.org/list.php?263}
\end{itemize}

% subsection reprap_org_forum (end)

\subsection{Suivi des anomalies} % (fold)
\label{sub:issue_tracker}

Si un bogue a \'et\'e d\'ecouvert dans le logiciel alors une question peut \^etre soulev\'ee dans le suivi d'anomalie (issue tracker) du projet.

\begin{itemize}
    \item \url{github.com/alexrj/Slic3r/issues}
\end{itemize}

\textbf{S'il vous pla\^it} prenez le temps de lire les questions d\'ej\`a poss\'ees pour voir si le probl\`eme a d\'ej\`a \'et\'e soumis. V\'erifiez \'egalement que le probl\`eme est un bogue dans l'application, des questions d'aise \`a l'utilisation ne doivent pas \^etre poss\'ees ici.

Si le bogue semble \^etre non d\'eclar\'ee alors s'il vous pla\^it lire le guide de rapport de bogue avant de soumettre: \url{https://github.com/alexrj/Slic3r/wiki/Quick-guide-to-writing-good-bug-reports}.

% subsection issue_tracker (end)

% section slic3r_support (end)